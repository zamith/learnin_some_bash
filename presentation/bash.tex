\documentclass[t,notes=show]{beamer}

\usetheme{Warsaw}

\usepackage[portuges]{babel}
\usepackage[utf8]{inputenc}
\usepackage[T1]{fontenc}
\usepackage{fancybox}
\usepackage{listings}
%\definecolor{keywords}{RGB}{255,0,90}
%\definecolor{comments}{RGB}{60,179,113}
%\lstset{language=Erlang,
%	    keywordstyle=color{keywords},
%	    commentstyle=color{comments}\emph}

\usepackage{listings}
\usepackage{color}
\definecolor{MyDarkBlue}{rgb}{0,0.08,0.45}
\definecolor{red}{rgb}{1,0,0}
\lstset{frame=shadowbox, rulesepcolor=\color{MyDarkBlue}, morecomment=[l]{\#},numbers=left, numbersep=7pt, language=bash, showstringspaces=false, basicstyle=\footnotesize}

% AMSLaTeX packages
%\usepackage{amsthm}
%\usepackage{amsmath}
%\usepackage{amsfonts}
%
%\theoremstyle{definition}
%\newtheorem{codelist}{Code Listing}[chapter]


% we want to use images
\usepackage{graphicx}

% PDF settings
\hypersetup{%
	pdftitle={BASH Scripting},%
	pdfauthor={Luís Ferreira},%
	pdfsubject={Bash Scripting},%
	pdfkeywords={Bash,Shell,Scripting}%
}

\title{BASH Scripting}
\author{Luís Ferreira \and André Santos \and Pedro Faria}
\institute[DI-UM]{
	Universidade do Minho \\ 
	CAOS Talks \\
	\vskip.5cm
	\texttt{zamith@groupbuddies.com\\
	andrefs@groupbuddies.com\\
	pedro.faria.80@gmail.com}
}

\begin{document}
	
	\frame[plain]{\titlepage}
	
	\begin{frame}[fragile]
		\frametitle{Pipes e redireccionamento}
		\begin{columns}
			\begin{column}[lc]{.5\textwidth}
				\begin{itemize}
					\uncover<1->{\item Operador $>$}
					\uncover<2->{\item Operador $>>$}
					\uncover<3->{\item Operador $<$}
					\uncover<4->{\item Pipe $|$}
					\uncover<5->{\item Redireccionar file descriptors}
					\uncover<6->{\item Auto-redireccionamento \emph{inline}}
				\end{itemize}	
			\end{column}

			\begin{column}[t]{.5\textwidth}
				\only<1>{
					\begin{semiverbatim}
						\begin{block}{Operador $>$}
							\$ echo Hello > testFile \\
							\$ cat testFile \\
							\emph{Hello}
						\end{block}	
					\end{semiverbatim}
				} 
				\only<2>{
					\begin{semiverbatim}
						\begin{block}{Operador $>>$}
							\$ echo Hello2 >> testFile \\
							\$ cat testFile \\
							\emph{Hello} \\
							\emph{Hello2} \\
						\end{block}	
					\end{semiverbatim}
				}
				\only<3>{
					\begin{semiverbatim}
						\begin{block}{Operador $<$}
							\$ cat < testFile \\ 
							\emph{Hello} \\
							\emph{Hello2} \\
						\end{block}	
					\end{semiverbatim}
				}
				\only<4>{
					\begin{semiverbatim}
						\begin{block}{Pipe $|$}
							\$ cat testFile | sort -r \\
							\emph{Hello2} \\
							\emph{Hello} \\
						\end{block}	
					\end{semiverbatim}
				}
				\only<5>{
					\begin{semiverbatim}
						\begin{block}{Redireccionar file descriptors}
							\$ rm direct/ \\
							\emph{rm: direct/: is a directory} \\
							
							\$ rm direct/ 2> error \\
							
							\$ rm direct/ 2>\&1   \\
							\emph{rm: direct/: is a directory} \\
						\end{block}	
					\end{semiverbatim}
				}
				\only<6>{
					\begin{semiverbatim}
						\begin{block}{HERE Documents}
							\$ bc << HERE \\
							> 12+12 -5 \\
							> 100*10-200 \\
							> HERE \\
							\emph{19} \\
							\emph{800}
						\end{block}	
					\end{semiverbatim}
				}
			\end{column}
		\end{columns}			
	\end{frame}
	
	\begin{frame}
		\frametitle{5 coisas que deves saber antes de escrever um script}
		\begin{enumerate}
			\item Há 3 tipos de quotes:
			\begin{description}
				\item[Aspas] Aceitas espaços e expande variáveis 
				\item[Pelicas] Aceita espaços mas não expande variáveis
				\item[Backslash] Faz escape ao valor de .
			\end{description}
			\item \emph{Globbing}
			\begin{itemize}
				\item *
				\item \{...\}
				\item outros (?, [\ldots], [\^{}\ldots])
			\end{itemize}
			\item \$(comando), encapsula o \emph{output} do comando
			\item var="ls ." \footnote{Não pode haver espaços entre o igual, e se houver espaços no valor este tem de estar entre aspas}
			\item Comentários são feitos com o \#
		\end{enumerate}
	\end{frame}
	
	\begin{frame}[fragile]
		\frametitle{O primeiro script}
		\begin{lstlisting}
			#!/bin/bash

			echo "Name of file to cat:"
			read file
			cat $file

			greeting="\nHello World"
			echo -e $greeting

			exit 0
		\end{lstlisting}
		\begin{itemize}
			\item A linha 1 diz à shell qual o interpretador a usar e começa sempre com \#!
			\item Por norma um script que termine correctamente deve retornar o valor 0
		\end{itemize}
		
		{\small Para correr o script basta dar permissões de execução a toda a gente (ou não), com \emph{chmod +x file.sh} e corrê-lo \emph{./file.sh}}
	\end{frame}		
	
	\begingroup
		\setbeamertemplate{background canvas}[vertical shading][top=white,bottom=green!25]
		\setbeamercolor{frametitle}{bg=green}
		\setbeamercolor{title in head/foot}{bg=green!75}		
	 
		\begin{frame}
			\frametitle{Comando do minuto}
		\end{frame}	
	\endgroup
	
	\begin{frame}
		\frametitle{Variáveis de ambiente}
		
		No início de qualquer script bash temos acesso a algumas variáveis, que podem ser alteradas com um export ou no .bashrc
		
		{\footnotesize
		\begin{description}
			\item[\$HOME] Home do utilizador actual
			\item[\$PATH] Lista de directorias onde são procurados os comandos
			\item[\$PS1] Definição da prompt (ex: \textbackslash h:\textbackslash W \textbackslash u \textbackslash \$)
			\item[\$PS2] Prompt secundária, normalemente $>$
			\item[\$IFS] \emph{Input Field Separator}. Lista de caracteres usados para separar palavras quando a ler do \emph{input}
			\item[\$0] O nome do script
			\item[\$\#] O número de parâmetros passados
			\item[\$\$] O PID do script (normalmente usado para criar ficheiros temporários) 
		\end{description}
		}
		
		Para ver todas as variáveis de ambiente, \emph{printenv}\footnote{\url{http://tldp.org/LDP/Bash-Beginners-Guide/html/sect_03_02.html}}
		
	\end{frame}	
	
	\begin{frame}
		\frametitle{Variáveis dos parâmetros}
		
		Este tipo de variáveis são um subconjunto das de ambiente, que dizem respeito aos parâmetros recebidos pelo script

		\begin{description}
			\item[\$1, \$2, ...] Os parâmetros passados, por ordem
			\item[\$*] Lista de todos os parâmetros, separadas pelo primeiro caracter em IFS
			\item[\$@] Uma variação de \$*, que usa sempre o espaço como separador (aconselhado)
		\end{description}
		
	\end{frame}	
		
\end{document}	
